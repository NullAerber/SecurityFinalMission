\section{目前研究现状}
机器学习方法能够基于大量数据进行自动化学习和训练,已经在图像、语音、自然语言处理等方面广泛应用。
\\\indent{}这些方法主要分为两类:大多数转化为有监督学习问题,则需要对数据进行标注,而其他的一些研究人员则试图以无监督的方式解决问题,例如通过异常检测技术,就不需要对数据进行标注。当可以得到标注数据时,有监督学习方法通常实现更强的泛化能力。然而在很多时候,我们很难获得精准的标注数据。在更多时候,我们可能只得到一小部分恶意URL和大量未标记的URL样本,缺乏足够可靠的负例样本。另一方面,如果我们简单地以无监督的方式解决它,那么已知恶意URL的标注信息就难以充分利用,可能无法达到令人满意的性能。
\\\indent{}机器学习应用于URL检测存在的最大的困难就是标签数据的缺乏。尽管有大量的正常访问流量数据,但恶意URL稀少,且变化多样,对模型的学习和训练造成困难。
\subsection{基于数理统计}
基于统计学习的web异常检测,通常需要对正常流量进行数值化的特征提取和分析。特征例如,URL参数个数、参数值长度的均值和方差、参数字符分布、URL的访问频率等等。接着,通过对大量样本进行特征分布统计,建立数学模型,进而通过统计学方法进行异常检测。
\subsection{基于单分类问题}
由于web入侵黑样本稀少,传统监督学习方法难以训练。基于白样本的异常检测,可以通过非监督或单分类模型进行样本学习,构造能够充分表达白样本的最小模型作为Profile,实现异常检测。
\subsection{基于文本分析(NLP)}
Web异常检测归根结底还是基于日志文本的分析,因而可以借鉴NLP中的一些方法思路,进行文本分析建模。这其中,比较成功的是基于隐马尔科夫模型(HMM)的参数值异常检测。