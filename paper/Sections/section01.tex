\section{引言}
近年来,随着移动网络环境的不断完善和智能手机的普及,互联网应用已深入渗透到生活的方方面面。CNNIC发布的报告[p]显示,截至2016年6月,中国网民规模达7.10亿,互联网的普及率达到51.7%,超过了全球平均水平3.1个百分点。万维网通常是人们接入互联网的主要入口,用户通过URL(统一资源定位符)直接或间接地获取互联网上的各类信息.与此同时,各种流氓网站、金融诈骗也藏身于此,URL鱼龙混杂。2015年,据卡巴斯基安全公报[p]统计,大约有196万台受感染的主机通过网上银行窃取用户账户,检测到的恶意对象包括脚本、漏洞和可执行文件等,数量高达1.2亿,通过恶意URL进行的攻击占整个网络攻击的73%以上。可以说,恶意URL是网络犯罪的基石。
\\\indent{}恶意URL在广义上是指用户非自愿访问的网站地址,这些网站内通常被植入了木马、病毒、广告等恶意代码,这些恶意代码通过伪装成正常服务来诱导用户进行访问.一旦进入这些恶意URL,用户通常会遭受广告弹窗、强制安装软件[p]或信息被盗等危害.每当用户面对陌生的URL时都需要一个安全的评估,因此对互联网的URL检测就显得十分必要.
\\\indent{}当前一些主流应用针对恶意URL的检测主要依靠黑名单[p]机制,如 Chrome浏览器的SafeBrowsing[p]和IE浏览器的SmartScreen[p]。然而,黑名单机制仅对名单内的恶意URL检测有效,不能识别名单以外的恶意URL,并且随着黑名单的不断增加,检测效率会不断下降.
\\\indent{}所以对于新出现的URL,基于机器学习分类算法的检测技术成为了另一种比较主流的识别方式[p]。机器学习检查技术识别恶意URL的主要流程如下:1)搜集训练样本;2)提取样本集中的URL异常特征并组成特征向量;3)根据样本集的特征向量训练出分类器模型;4)提取待检测URL特征,并将特征代入分类器模型中进行分类。基于机器学习分类算法的恶意URL检测技术的检测准确率依赖于异常特征的提取和算法本身的特点,每一种单独的分类算法都有自己擅长处理的数据和不擅长处理的数据,对于同一组数据,有些分类器擅长,有些分类器却不擅长。本项目主要是针对不同的方法进行相关的实践、分析和总结,最终提出自己对于恶意URL的改进和设计。
\\\indent{}就机器学习在恶意URL上的检测,比较出名的是山石网科在Blackhat Asia(亚洲黑帽大会)上发表的论文[p]“Beyond The Blacklists: Detecting Malicious URLS Through Machine Learning”,率先采用机器学习的方式进行侦查。该基于大数据内已知的恶意软件样本,我们可以提炼出简洁的功能模型,代表许多不同的恶意软件中常用的相似性连接行为。这个模型可以用于检测有着共同的网络特征的未知恶意软件变种。
\lhead{2018年信息安全课程设计}
\chead{基于机器学习的恶意URL检测}
\rhead{2015级信息安全 周孟莹}
\lfoot{\zihao{-5}\cop}
