\section{结论}
随着科学技术的飞速发展,研究的难度将不断地提升,研究的领域将不断地扩大,多学科的交叉研究势在必行。将可拓集的研究与模糊集和粗集的研究相结合,将可能对人工智能、信息与控制、经济管理等多个领域产生推动作用。2005年12月6 日-7日,我国著名的香山科学会议在北京香山召开,其中第271次学术讨论会,以“可拓学的科学意义与未来发展”为主题,进行了2天的讨论,由此将可拓学提升到一个新的台面上来了。我们可以类似于模糊集引进模糊凸集和模糊凸函数那样\cite{ref07},引进可拓凸集和可拓凸函数\cite{ref05}\cite{ref06},在此基础上提出可拓最优化问题。这样首先在工程和投资决策中实现\cite{ref03}\cite{ref04}。我们要通过研究刘晏的成功方法,利用可拓的理论和转换桥的思想(即利用“各行其道,各得其所”的思想),尽快编成计算机程序,用电脑出谋划策,努力探索“出点子”、“想办法”的规律。尽快设计制造出模拟人脑思维的机器人,使今天的设想成为明日的现实,以此造福于人类。
