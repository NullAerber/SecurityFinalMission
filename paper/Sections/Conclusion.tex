\section{总结}
\subsection{基于二分类的机器学习}
逻辑回归判断和svm都是常见的机器学习分类算法,从目标函数来看,区别在于lg采用的是logistical loss,svm采用的是hinge loss。这两个损失函数的目的都是增加对分类影响较大的数据点的权重,减少与分类关系较小的数据点的权重。两者的根本目的都是一样。因此在很多实验中,两种算法的结果是很接近的。

SVM的处理方法是只考虑支持向量,也就是和分类最相关的少数点去学习,对数据的异常点(离群值)比较敏感,而且输入svm的数据维度不宜太高,这就是为什么本文在第四部分中对数据进行聚类降维后SVM的效果会明显提升的原因。但是逻辑回归是通过非线性映射,大大减小了离分类平面较远的点的权重,相对提升了与分类最相关的数据点的权重。


逻辑回归相对来说模型更简单,更容易理解,实现起来,特别是大规模线性分类时比较方便。而SVM的理解和优化相对来说复杂一些,有一套结构化风险最小化的理论基础。
\subsection{基于LSTM序列模型分类器}
对于LSTM序列模型分类器来说,由于输入是json格式的数据,在对json格式的数据分类上还是比较优良的,在json模拟数据集上可以达到95\%左右的精确度,这说明这个模型确实有效。不过这些数据在模拟数据集(Mock API)上生成,生成的log数据可能有一定偏向性,测试数据集可能和训练数据集风格过于一致,所以精确度很高。以至于当使用普通的纯URL进行测试时,精度降了很多。可能也是因为样本已经完全不在一个分布,数据的格式也不太相同的原因。