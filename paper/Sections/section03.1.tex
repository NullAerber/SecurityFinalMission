\section{进行物元变换}
刘晏巧解中原“艰食”之围的方法,就是通常人们所指的‘窍门’、‘点子’、‘办法’和‘策略’。这些方法,为什么刘晏能够想到?它有否规律可行,有无方法可依?如果有,用计算机去实现这些,正是人类梦寐以求的,这也是本世纪和下世纪人类研究的重大课题。实际上,刘晏的方法是采用了以下四种物元变换:
\begin{itemize}
    \item \textbf{置换变换:}\\用某一事物代替另一事物,用某一特征代替另一特征,用某一量值代替另一量值的变换,称为置换变换。\\刘晏把余粮区的$x$万吨粮食先运到中原地区,以暂缓中原“艰食”的矛盾,就是灵活地应用了这一变换。
        \item \textbf{增删变换:}\\根据目标市场的需要,确定对所收购物质的增、减量,称此类变换为增删变换。
\\刘晏统筹全局,分别考虑价格和运输两个因素,确定增、减各地区的粮食购买量,以达到节省运费及开支,多购粮之目的。
                \item \textbf{组分变换:}\\对一种物质产品分解成不同的部分,或将多种物质产品通过组合后形成一种新物质产品的变换,称为组分变换。
\\刘晏成功地运用组分变换中的分解变换,利用经济统计数据,把各州、县几十年内的粮价与购粮数量各分成五等,由两者间的数量关系,归纳出价格与购量成反比的趋向,由此制定出合理的购粮规则。
                        \item \textbf{扩缩变换:}\\此即对物元进行扩大或缩小的变换。
\\刘晏的购粮安排可用以表\ref{table01}表示:
\end{itemize}
\begin{center}
\captionof{table}{$A_i$州、县购粮统计表}
\label{table01}
\begin{tabular}{|c|c|c|c|c|c|}
   \hline
   $\vcenter{\hbox{%
     \begin{tikzpicture}
       \path[use as bounding box] (\tabcolsep,0) rectangle (3.5cm-\tabcolsep,1.2cm);
        \draw (0cm,1.2cm) -- node[auto]{\zihao{-6}购量} node[auto,swap]{} (3.5cm,0cm);
        \draw (2.2cm,1.2cm) -- node[auto]{\zihao{-6}价格} node[auto,swap]{} (3.5cm,0);
        \draw (0cm,0cm) -- node[auto]{\zihao{-6}收购等级} node[auto,swap]{} (3.5cm,0cm);
    \end{tikzpicture}%
    }}$&\zihao{6}$P_1$&\zihao{6}$P_2$&\zihao{6}$P_3$&\zihao{6}$P_4$&\zihao{6}$P_5$\\
   \hline
   \zihao{6}1&&\zihao{6}$C_4$&&&\\
\hline
\zihao{6}2&\zihao{6}$C_5$&&&&\\
\hline
\zihao{6}3&&&&\zihao{6}$C_2$&\\
\hline
\zihao{6}4&&&\zihao{6}$C_3$&&\\
\hline
\zihao{6}5&&&&&\zihao{6}$C_1$\\
\hline
\end{tabular}
\end{center}

\indent{}从统计表1知,价格与购量关系可能呈非线性关系(见下图1中的实曲线)。
\begin{figure}[!h]
\centering
 \includegraphics[scale=0.4]{Figs/01.png}
\caption{价格与购量关系}
\label{fig:01}
\end{figure}
\\\indent{}通过扩缩变换,将非线性关系化为线性供给关系,(见图\ref{fig:01}中的虚直线),从而大大地简化了运算。
