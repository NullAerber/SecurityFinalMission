\mtitle{基于机器学习的恶意URL检测}
\mauthorinfo{周孟莹}{兰州大学 信息科学与工程学院 信息安全\\甘肃,兰州 730000}
\paragraph{\textbf{摘 要:}随着互联网的普及使用,越来越新奇的恶意URL攻击出现在网络上,采用黑名单机制的方法已经不能提供完全的保护。对于这些URL,基于机器学习分类算法的检测技术成为了另一种比较主流的方式。本文通过对Logical Regression逻辑回归,SVM支持向量机和LSTM序列神经网络三种方式,对一系列样本数据进行测试,得出每种模型下的结果,并整体对其进行分析和改进。由于本课题是工程课题,因此本文主要记录了实验过程中的相关信息和数据结果,涉及到的数理理论分析会比较少。}
\paragraph{\textbf{关键词:}恶意URL检测;逻辑回归;SVM;LSTM}
% \paragraph{图法分类号:TJ 399;O 343.3\quad\quad 文献标志码:A}
\mtitle{A Machine Learning System for Potential Malicious URL Detection}
\mauthorinfo{Aerber Zhou}{School of Information Science and Engineering, Lanzhou University, Security Information\\Gansu Lanzhou P.R.China,510006}
\paragraph{\textbf{Abstract:} With the popularity of the Internet, more and more novel malicious URL attacks appear on the network, and the method of blacklisting mechanism can not provide complete protection. For these URL, detection technology based on machine learning classification algorithm has become another mainstream method. In this paper, through three ways of Logical Regression logic regression, SVM support vector machine and LSTM sequence neural network, a series of sample data are tested, and the results of each model are obtained, and the overall analysis and improvement are carried out. Because this topic is a project topic, this article mainly recorded the relevant information and data results in the experiment process.}
\paragraph{\textbf{Keywords:} URL Attack Detection, Logical Regression, Support Vector Machine, Long Short Term}
\paragraph{} 