\mtitle{基于机器学习的恶意URL检测}
\mauthorinfo{周孟莹}{兰州大学 信息科学与工程学院 信息安全\\甘肃,兰州 730000}
\paragraph{\textbf{摘 要:}从分析唐代刘晏以聪明才智巧解中原“艰食”之围的实例,指出刘晏应用的方法,实质上就是一种可拓运筹决策的思想,所用的变换工具,可归纳为物元变换。因此,用物元变换解决难题,古已有之。认真地研究诸如此类的问题,对于探索出“点子”,想“办法”的规律,无疑地是有益的。}
\paragraph{\textbf{关键词:}可拓;运筹;物元变换}
% \paragraph{图法分类号:TJ 399;O 343.3\quad\quad 文献标志码:A}
\mtitle{A Machine Learning System for Potential Malicious URL Detection}
\mauthorinfo{Bing-Yuan Cao}{School of Mathematics and Information Sciences, Guangzhou University,\\
Key Laboratory of Mathematics and Interdisciplinary Sciences of Guangdong Higher EducatioInstitutes,\\ Guangzhou University 510006  P.R.China}
\paragraph{\textbf{Abstract:} By analyzing an instance of Liu Yan’s idea in the Tang Dynasty, the wisdom solution to "storm water"  in heartland surrounding, the writers in the paper points out that Liu Yan's method of application, in essence, is an idea of decision making through operations research in extension, and the use of  transformation can be summarized as a matter-element one. Thus, the matter-element transformation to problems existed even since ancient times. A serious look at issues like these, for the law of finding out "ideas", and coming up with "approaches", no doubt, becomes useful.}
\paragraph{\textbf{Keywords:} Extension, operation research, matter-element transformation}
\paragraph{} 