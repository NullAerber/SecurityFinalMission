\section{进行相容性检验}
令
\begin{displaymath}
\Delta C_i=\frac{M-M^*}{P_i},\qquad\qquad(\mbox{当}M>M^*),
\end{displaymath}
\indent{}或
\begin{displaymath}
\Delta C_i^{'}=\frac{M-M^*}{F_i},\qquad\qquad(\mbox{当}M>M^*),
\end{displaymath}
其中$M$—总资金量;$M^*$—呈报的购粮费;$P_i$—表示州、县的粮价;$\Delta C_i$—粮价低的州、县的增购量;$\Delta C_i^{'}$— 运距短的州、县的增购量。
\indent{}判定:
\\\indent{}若$\Delta C_i>0$,应增加订购量。若$\Delta C_i<0$,应减少甚至不购粮。
\\\indent{}当$\Delta C_i>0$时,可根据$\Delta C_i$的大小来确定粮食增购量的多少。
\\\indent{}通过相容性的检验,也证实了上述变换措施的合理性和可行性。
\\\indent{}刘晏就是用上述的可拓运筹决策方法,迅速、经济、圆满地解决了中原“艰食”之围,受到了世人的敬仰。
\\\indent{}事实上,根据《梦溪笔谈》记载\cite{ref04},刘晏是采取了如下三项措施:
\\\indent{}1、积极整顿漕运;
\\\indent{}2、合理组织运输,采取“因地制宜、分段运输”的办法,费用开销由逐级核算到汇总核算;
\\\indent{}3、改进运输包装,改散装为袋装,大大减少了损耗。
\\\indent{}由于采取这些措施,不仅加快了运输速度,而且极大地减少了运输费用。例如:过去由扬州运粮至长安需要花9个月的时间,沿途的损耗高达$20\%$。经刘晏改进后仅需40天时间,且无半斗损失,每石米只需要700文的运杂费,“人以为神”,从而使长安粮价平稳,唐肃宗曾称他为“当朝的萧何”。
\\\indent{}刘晏此举充分反映了我们的祖先的聪明才智,体现了我们中华民族的灿烂文化。如果能将他们的经验模拟下来,发扬光大,将是多么好的事情。本文告诉我们,这种“聪明”、“智慧”、“解难题”一类问题,是有规律可循的;巧计与方法,是有理论可依的。可拓运筹决策就是解决这一类不相容问题的有效方法。因此,物元分析中的可拓运筹决策方法,不仅渊源已久,而且来源于实践,它必能指导实践,具有强大的生命力。